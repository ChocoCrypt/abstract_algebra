% article example for classicthesis.sty
\documentclass[10pt,a4paper]{article} % KOMA-Script article scrartcl
\usepackage{import}
\usepackage{xifthen}
\usepackage{pdfpages}
\usepackage{transparent}
\newcommand{\incfig}[1]{%
    \def\svgwidth{\columnwidth}
    \import{./figures/}{#1.pdf_tex}
}
\usepackage{lipsum}     %lorem ipsum text
\usepackage{titlesec}   %Section settings
\usepackage{titling}    %Title settings
\usepackage[margin=10em]{geometry}  %Adjusting margins
\usepackage{setspace}
\usepackage{listings}
\usepackage{amsmath}    %Display equations options
\usepackage{amssymb}    %More symbols
\usepackage{xcolor}     %Color settings
\usepackage{pagecolor}
\usepackage{mdframed}
\usepackage[spanish]{babel}
\usepackage[utf8]{inputenc}
\usepackage{longtable}
\usepackage{multicol}
\usepackage{graphicx}
\graphicspath{ {./Images/} }
\setlength{\columnsep}{1cm}



\begin{document}
    %========================{TITLE}====================%
    \title{{  Respuestas parcial Algebra abstracta  }}
    \author{{Rodrigo Castillo}}
    \date{\today}

    \maketitle


    %=======================NOTES GOES HERE===================%
    \section{Punto 1:Codigo de Reed Solomon}
    % ====|ACA EMPIEZA EL PUNTO 1-a|====
    escriba el vector $x_i = \alpha ^{i-1} $ que define el código C
    \\
    el vector es $v = 3 ^{0} , 3 ^{1} , ... , 3 ^{6}   $ pero en $F[7]$
    \\
    luego es $ vec = [ 1, 3 ,2 , 6 , 4, 5, 1]$

    % ====|ACA TERMINA EL PUNTO 1-a|==== %
    la matriz generadora del código es asumiendo que $k = 3$ es... :
    \\
    la primera fila es 1 , la segunda es el vector $vec$ y la tercera son los
    elementos de vec elevados al cuadrado en congruencia $mod(7)$
    \begin{equation}
        \begin{pmatrix}
            1 & 1 & 1 & 1 & 1 & 1
            \\
            1 & 3 & 2 & 6 & 4 & 5
            \\
            1 & 2 & 4 & 1 & 2 & 4
        \end{pmatrix}
    \end{equation}
    \\
    para encontrar la distancia partiremos de la identidad $d = n-k +1$, por lo
    que $6-3+1 = 4$ , luego $d = 4$
    \\
    $c = \frac{d-1}{2} = 1$
    \\
    $L_0 = 4$
    \\
    $L_i = 2$
    \\
    \textbf{decodificar [2,6,0,5,1,3]}
    \begin{equation}
        \begin{pmatrix}
            1 & 1 & 1 & 1 & 1 & 2 & 2 & 2
            \\
            1 & 3 & 2 & 6 & 4 & 6 & 4 & 5
            \\
            1 & 2 & 4 & 1 & 2 & 0 & 0 & 0
            \\
            1 & 6 & 1 & 6 & 1 & 5 & 2 & 5
            \\
            1 & 4 & 2 & 1 & 4 & 1 & 4 & 2
            \\
            1 & 5 & 4 & 6 & 2 & 3 & 3 & 6
        \end{pmatrix}
        \cdot
        \begin{pmatrix}
          Q_{0,0}
          \\
          Q_{0,1}
          \\
          Q_{0,2}
          \\
          Q_{0,3}
          \\
          Q_{0,4}
          \\
          Q_{1,0}
          \\
          Q_{1,1}
          \\
          Q_{1,2}
        \end{pmatrix}
        = 0
    \end{equation}
    por lo que tenemos que
    \\
    $Q_{00} =  0 $
    \\
    $Q_{01}  + 3Q_{11} + 6Q_{12} = 0$
    \\
    $Q_{02} + 2Q_{11} = 0$
    \\
    $Q_{03} + 2Q_{12} = 0$
    \\
    $Q_{04} = 0$
    \\
    $Q_{11} = 2Q_{31} + 4Q_{12} = 0$
    \\
    de lo anterior podemos concluir que
    \\
    $Q_{01} = -3$ , $Q_{03} = 0 ,  Q_{10} = 2 , Q_{02} = 2$ y además $Q_{00} =
    0Q, Q_{04} = 0$
    \\
    \textbf{para el punto f tengo que evaluar el código en la función del punto e}




    \section{Punto 2: Demuestre que cualquier ideal en $Z[i]$ distinto del
    ideal 0 contiene un entero}
    \textbf{demostración por contradicción}
    sea $I$ un ideal en $Z[i]$ tal que $I$ no contiene ningún entero, sea $i$
    un elemento de $I$ y $a$ un elemento de $Z[i]$.
    \\
    Note que $i \cdot a \in I$ puesto que $I$ es un ideal, pero como $I$ no
    contiene enteros, entonces tenemos que $i \cdot a \notin Z[i]$ , por lo que
    $I \notin Z[i]$ y esto es una contradicción , por lo tanto $I$ contiene al
    menos un entero.





    \section{sea G un grupo cualquiera y sea $G' = \{xyx ^{-1} y ^{-1} ,  x , y
    \in G   \}$}
    \textbf{A}
    demuestre que $G'$ es un subgrupo normal de $G$
    % ====|ACA EMPIEZA EL PUNTO 3-a|====
    sea $h \in G'$ y $s \in G$ , por lo que, aplicando el test, tengo que ...
    \\
    $h \cdot  xyx ^{-1} y ^{-1} \cdot h ^{-1}   $ es de la forma $hxyx ^{-1}  y
    ^{-1}  h ^{-1} $ , ahora , podemos ver que $hsh ^{-1}  \in G'$ por lo que
    $G'$ \textbf{es un grupo normal de $H$}



    % ====|ACA TERMINA EL PUNTO 3-a|==== %
    \textbf{b}
    Demuestre  que $G/G'$ es un grupo abeliano()
    \\
    para pobar que $G/G'$ es un grupo abeliano, tomemos dos elementos $a,b$
    tales que $a,b \in G/G'$ , por lo que tengo que probar que $aba ^{-1} b
    ^{-1}  = 1_{G/G'}$ , por lo que
    \\
    $ac a ^{-1} c ^{-1} \cdot  bdb ^{-1} c ^{-1}  = acbda ^{-1} c ^{-1} d ^{-1}
    d ^{-1}  = 1_{G/G'}     $
    % ====|ACA EMPIEZA EL PUNTO 3-b|====



    % ====|ACA TERMINA EL PUNTO 3-b|==== %
    \section{Punto4 :  sea $f(x)$ = $x ^{3}  + x + 1 $ in $Z_7[x]$}
    \textbf{a: el polinomio es irreducible}
    \\
    este punto salió en los ejercicios del teorema de galois .
    \\
    \textbf{Demostración por contradicción:} supongamos que el polinomio $f(x)
    = x ^{3} + x + 1  $ es reducible, es decir que existen dos polinomios $g y
    g' $ tales que $g \cdot g' = x ^{3} + x + 1  $
    \\
    estos polinomios deben ser de grados $2 $ y $1$ , de lo contrario, no es
    posible obtener un polinomio de grado $3$ como producto de la multiplicación de polinomios
    diferentes a esos grados ,
    \\
    Sea $g(x)$ un polinomio cualquiera de grado 2, luego $g(x)$ es de la forma
    $n_1 x ^{2} + n_2 x ^{1} + n_3 k  $ en donde $n_1 \not= 0$ , ahora, $g'(x)
    = m_1 x + 0 $ donde $m_1 \not= 0$
    \\
    como $n_1 , m_1 \not= 0$ , por lo que la multiplicación de ambos polinomios
    contendrá al menos un elemento elevado al cuadrado. por lo que es imposible
    que su multiplicación sea $ x ^{3} +x + 1 $ pues este polinomio no contiene
    elementos elevados al cuadrado.























    %=======================NOTES ENDS HERE===================%

    % bib stuff
    \nocite{*}
    \addtocontents{toc}{{}}
    \addcontentsline{toc}{section}{\refname}
    \bibliographystyle{plain}
    \bibliography{../Bibliography}
\end{document}
