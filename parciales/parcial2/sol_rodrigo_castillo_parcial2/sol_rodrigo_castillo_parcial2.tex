% article example for classicthesis.sty
\documentclass[10pt,a4paper]{article} % KOMA-Script article scrartcl
\usepackage{import}
\usepackage{xifthen}
\usepackage{pdfpages}
\usepackage{transparent}
\newcommand{\incfig}[1]{%
    \def\svgwidth{\columnwidth}
    \import{./figures/}{#1.pdf_tex}
}
\usepackage{lipsum}     %lorem ipsum text
\usepackage{titlesec}   %Section settings
\usepackage{titling}    %Title settings
\usepackage[margin=10em]{geometry}  %Adjusting margins
\usepackage{setspace}
\usepackage{listings}
\usepackage{amsmath}    %Display equations options
\usepackage{amssymb}    %More symbols
\usepackage{xcolor}     %Color settings
\usepackage{pagecolor}
\usepackage{mdframed}
\usepackage[spanish]{babel}
\usepackage[utf8]{inputenc}
\usepackage{longtable}
\usepackage{multicol}
\usepackage{graphicx}
\graphicspath{ {./Images/} }
\setlength{\columnsep}{1cm}

% ====| color de la pagina y del fondo |==== %
\pagecolor{white}
\color{black}



\begin{document}
    %========================{TITLE}====================%
    \title{{  Parcial 2 algebra abstracta  }}
    \author{{Rodrigo Castillo}}
    \date{\today}

    \maketitle


    %=======================NOTES GOES HERE===================%
    \section{Punto1 construya un campo con $7 ^{3} $elementos}
        \textbf{solucion}
        \\ el Teorema de Galois visto en clase nos dice que un polinomio de
        grado 3 en $Z_p$nos genera un campo de $7 ^{3} $elementos.
        \\
        sea el polinomio : $x ^{3}  +x ^{2} +  x $ $\in Z_7$ tenemos que genera
        un campo de $7 ^{3} $ elementos
        \\
        \textbf{el polinomio es irreducible}
        \\
        supongamos que existen dos polinomios $x,y$ tales que $x \cdot y = x
        ^{3}  + x ^{2}  + x$ .  por lo tanto, $x$ es de grado 2 y $y$ es de
        grado 1 (o el caso análogo). luego x es de la forma $nx ^{2}  + nx $ y
        y es de la forma $nx$ , luego $x \cdot y$ es un polinomio de laforma
        $nx ^{3} + 2nx \not= x ^{3} + x ^{2}  + x  $.

        \section{Punto 2:  Sea G un grupo y $g \in G$ , demuestre que si
        ${e,g}$ es un grupo normal entonces $g$ pertenece al centro de $G$}

        supongamos que $g \in G$  y que $g \not= e$ , además, supongamos que
        ${e,g}$ es un grupo normal, por lo tanto, para cualquier $h \in G$ se
        tiene que $hegh ^{-1} $ =$eg$ , como $eg = g$, se tiene que $ hgh ^{-1}
        = g$ , por lo tanto se tiene que $gh = hg$, por lo tanto $g \in Z(G)$


        \section{Punto 3: demuestre que el grupo Klein V4 es isomorfo a $Z_2 \times Z_2$}
        \textbf{aclaración:} , definí los elementos como $a,b$ , sin embargo, $a,b$
        deberían ser tomados en $Z_2$ como $a = 0 , b=1$

            \subsection{existe una función biyectiva del conjunto $KleinV4$
            hasta $ Z_2 \times Z_2$}
                sea $F$ una función desde $K_4$ hasta $Z_2 \times Z_2$ tal que a cada elemento de $K_4$ que es de tipo:
                \subsubsection{inyectividad}
                    $\begin{pmatrix}
                      a & b
                      \\
                      -b & a
                    \end{pmatrix}$
                    le asigna un elemento $(a,b) , (b,a)$
                    por lo tanto para cada $k \in 4_4$ existe un $z \in Z_2
                    \times Z_2$ tal que $f(k) = z$
                \subsubsection{sobreyectividad}
                supongamos que $f(k) = f(k')$ , así se tiene que $(a,b) , (b,a)$ = $(a',b') , (b',a')$ luego $ k = \begin{pmatrix}
                    a & b
                    \\
                    -b & -a
                \end{pmatrix}
                $ y $k ' = \begin{pmatrix}
                    a' & b'
                    \\
                    -b & -a'
                \end{pmatrix}
                $ y así obtenemos que $k = k'$
                por lo tanto $f$es sobre
                \subsection{la imagen de la operación es la operación de las imágines}
                esto es facil de ver en $Z_2 \times Z_2$ pues es un conjunto de 4 elementos


        \section{ Punto 4: Sea A un anillo conmutativo ....}
            sea $a \in A$ tal que $a \not= 0$ y que $a$ no es divisor de $0$.
            \\ como $A$ es un conjunto finito y además como $a$ no es divisor
            de 0, puedo elevar a $a$ por todos los elementos de $A$ sabiendo
            que ninguno de ellos va  aser divisor de 0, por lo que tengo un
            anillo de $[a , a ^{2} ... , a ^{n+1}  ]$ , por lo tanto... $1 = a
            ^{n +1} = ax ^{n}  $
             por lo tanto existe $n \in A$ tal que $a ^{n}  = 1$. luego la
             inversa de $a$ es $a ^{n+1} $




























    %=======================NOTES ENDS HERE===================%

    % bib stuff
    \nocite{*}
    \addtocontents{toc}{{}}
    \addcontentsline{toc}{section}{\refname}
    \bibliographystyle{plain}
    \bibliography{../Bibliography}
\end{document}
