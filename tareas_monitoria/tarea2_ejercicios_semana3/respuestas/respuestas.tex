article example for classicthesis.sty
\documentclass[10pt,a4paper]{article} % KOMA-Script article scrartcl
\usepackage{import}
\usepackage{xifthen}
\usepackage{pdfpages}
\usepackage{transparent}
\newcommand{\incfig}[1]{%
    \def\svgwidth{\columnwidth}
    \import{./figures/}{#1.pdf_tex}
}
\usepackage{lipsum}     %lorem ipsum text
\usepackage{titlesec}   %Section settings
\usepackage{titling}    %Title settings
\usepackage[margin=10em]{geometry}  %Adjusting margins
\usepackage{setspace}
\usepackage{listings}
\usepackage{amsmath}    %Display equations options
\usepackage{amssymb}    %More symbols
\usepackage{xcolor}     %Color settings
\usepackage{pagecolor}
\usepackage{mdframed}
\usepackage[spanish]{babel}
\usepackage[utf8]{inputenc}
\usepackage{longtable}
\usepackage{multicol}
\usepackage{graphicx}
\graphicspath{ {./Images/} }
\setlength{\columnsep}{1cm}

% ====| color de la pagina y del fondo |==== %
\pagecolor{white}
\color{black}



\begin{document}
    %========================{TITLE}====================%
    \title{\rmfamily\normalfont\spacedallcaps{ Ejercicios semana 3 }}
    \author{\spacedlowsmallcaps{Rodrigo Castillo}}
    \date{\today}

    \maketitle


     % ====| Loguito |==== %
    \includegraphics[width=0.1\linewidth]{negro_cara.png}
    %=======================NOTES GOES HERE===================%
    \section{escriba el algoritmo para encontrar el maximo comun divisor $
    gcd(a,b)  $ y para escribir $ ax + by  $ con $a,b \in Z $ }

    \begin{lstlisting}[language=Python]
        gcd(a,b):
    if(b==0):
        return a
    else:
        return gcd(b, amodb )
    \end{lstlisting}


    \section{Sea $ m > 2  $  un entero y consideramos $ Z_m  $  . Demuestre que la suma en $ Z_m  $
    definida como $ [x]_m + [y]_m = [x + y]_m  $  es bien definida, es decir no depende
    de los representantes elegidos para calcularla}
        \subsection{demostracion}
            sean $ x_m , y_m \in Z_m  $ .
                \subsubsection{caso base}
                sea $ m = 2  $ , por lo tanto $ x_m + y_m  $  = $ [x+y]_m  $
                \subsubsection{Inducción}
                supongamos que $ m  $ es un entero mayor que 2 y que $ x_m +
                y_m = [x+y]_  $ está bien definido, por lo tanto , sin pérdida
                de generalidad , $ x_{m+1} + y_{m+1} = [x+y]_m +2 = [x+y]_{m+1}
                $.


    \section{Teorema chino del resto:}
        \subsection{}
        $ 6x \equiv 8(mod10)  $
        \\$ 9x \equiv 15(mod21)  $
        \\ para resolver esto, hay que hacer uso del teorema chino del resto :

        \subsection{Uso del teorema chino del resto}
        lo primero que debemos saber es que el $ gcd(10,21)  $sea diferente de 1 para ver si tiene solución:
        \\ luego de calcularlo, sé que $ gcd(10,21) =1   $ , por lo tanto el sistema tiene solución, ahora...
        \\
        $ 6x = 8 + 10k  $
        \\ $ 9y = 15 + 21p  $
        \\ ahora...
        \\ $ 6x \equiv 15(mod21)  $
        \\ $ 48 + 60k = 15(mod21)  $
        \\ $ 60k \equiv -33(mod21)  $
        \\ $ 60k = 1 mod(12)  $
        \\ $ k = 1 mod(12)  $
        \\

    \section{}

    \section{Sea $ R  $  un anillo cualquiera. Demuestre que el conjunto $ M_2
    (R)  $  de las matrices $ 2 \times  2  $  a coeficientes en $ R  $  es un
    anillo}
        \subsection{Suma}
            sean $ a,b,c,d \in R  $ , $ z,x,c,v \in R  $  por lo tanto:
                \begin{equation}
                    \begin{pmatrix}
                        a & b
                        \\ c & d
                    \end{pmatrix}
                    +
                    \begin{pmatrix}
                        z & x
                        \\ m & v
                    \end{pmatrix}
                    =
                    \begin{pmatrix}
                        a+z & b+x
                        \\c+m & d+v
                    \end{pmatrix}

                \end{equation}
                también es una matriz $ 2 \times 2  $ , además, como $
                a,b,c,d,z,x,m,v \in R  $ entonces $ a+z \in R , b+x \in R...
                $ , análogamente pasa lo mismo con la resta , la multiplicación
                , y todas las propiedades de R.














    %=======================NOTES ENDS HERE===================%

    % bib stuff
    \nocite{*}
    \addtocontents{toc}{\protect\vspace{\beforebibskip}}
    \addcontentsline{toc}{section}{\refname}
    \bibliographystyle{plain}
    \bibliography{../Bibliography}
\end{document}
