% article example for classicthesis.sty
\documentclass[10pt,a4paper]{article} % KOMA-Script article scrartcl
\usepackage{lipsum}     %lorem ipsum text
\usepackage{titlesec}   %Section settings
\usepackage{titling}    %Title settings
\usepackage[margin=10em]{geometry}  %Adjusting margins
\usepackage{setspace}
\usepackage{listings}
\usepackage{amsmath}    %Display equations options
\usepackage{amssymb}    %More symbols
\usepackage{xcolor}     %Color settings
\usepackage{pagecolor}
\usepackage{mdframed}
\usepackage[spanish]{babel}
\usepackage[utf8]{inputenc}
\usepackage{longtable}
\usepackage{multicol}
\usepackage{graphicx}
\graphicspath{ {./Images/} }
\setlength{\columnsep}{1cm}

% ====| color de la pagina y del fondo |==== %
\pagecolor{black}
\color{white}
\begin{document}
    %========================{TITLE}====================%
    \title{\rmfamily\normalfont\spacedallcaps{ Ejercicos monitoría semana 4 }}
    \author{\spacedlowsmallcaps{Rodrigo Castillo}}
    \date{\today}

    \maketitle

     % ====| Loguito |==== %
    \includegraphics[width=0.1\linewidth]{negro_cara.png}
    %=======================NOTES GOES HERE===================%
    % ====| PUNTO 1 |==== %
    \section{Sea $ R  $ un anillo, demuestre que }
        \subsection{si existe la identidad 1 es única}
         Demostracion:

         \\ Supongamos que existen dos indentidades llamados$ 1  $  y $ 1'  $.
         \\ por lo tanto para todo $ a \in  R  $ se tiene que $ a \times 1 = a   $
         \\ tambien se tiene que $ a \times 1' = a  $ , por lo tanto ...
         \\ $ a \times 1 = a  $
         \\ $ a \times 1' = a  $
         \\ y por transitividad...
         \\ $ 1 = 1'  $
         \\ por lo tanto solamente existe un inverso multiplicativo

        \subsection{si un elemento $ a  $ tiene inverso multiplicativo este es
        único}

            Supongamos que un elemento $ a \in R  $ tiene múltiples inversos
            multiplicativos,  es decir que para todo $ a \in R  $ existen $ b,
            b' \in R$ tales que $ ab = 1  $ y $ ab' =1 $. por lo tanto
            \\ $ b = \frac{1}{a}  $
            \\ $ b' = \frac{1}{a}  $
            \\ $ b = b'  $
            \\ luego el inverso multiplicativo en un anillo es único
            \\ (no sé que tanto sentido tenga esta demostración jaja, preguntar)

    % ====| PUNTO 2 |==== %
    \section{Demuestre que los siguientes son subanillos de C}
        \subsection{Los enteros de Gauss}
            \begin{equation}
                Z[i] = \{a + bi ,a,b \in Z \}
            \end{equation}

            % ====| DEMOSTRACION |==== %
            Demostracion :
            \\ sean $ c , c' \in  Z[i] $  , luego $ c - c'  $ es de la forma
            \\ $ a-a' + bb'i a,a' , b,b' \in Z  $  , por lo tanto, note que $
            a+a' \in Z  $  y que $ bb' \in Z0$ , por lo tanto $ c,c' \in Z  $
            \\
            de forma análoga, se tiene que $ c*c' \in Z  $  , luego $ Z[i]  $
            es un subanillo de $ R  $ .

            % ====| ACA TERMINA LA DEMOSTRACION DE LOS ANILLO |==== %

        \subsection{Los enteros de einstein}
        \begin{equation}
                Z[w] = \{a + bw ,a,b \in Z \}
        \end{equation}
        donde $ w = e ^{ \frac{2 \pi i}{3} } = \frac{-1 + \sqrt{3} }{3}  $
            % ====| Demostracion empieza aca |==== %
            \subsubsection{Demostracion}
              tengo que $ w = e ^{ \frac{2 \pi i}{3} } = \frac{-1 + \sqrt{3} }{3}  $
              \\ sean $ n,m \in Z[w]  $ luego $ n-m  $  = $ a_1 -a_2 - b_1 -
              b_2 w  $
              \\ por lo tanto , claramente,  $ a_1 - a_2 \in Z  $  , sin
              embargo falta ver que $ b_1 - b_2 -2w \in Z  $ , por lo tanto ...

              \\$ b_1 b_2 - 2( e ^{ \frac{2 \pi i}{3} } = \frac{-1 + \sqrt{3} }{3})  $
              \\ note que $ b \in Z  $ , por lo tanto la resta está bien
              definida por lo tanto la expresion anterior pertenece a $ Z  $  $
              \\

    \section{Demuestre que si $ \theta : R \to S  $ es un homomorfismo
    invertible de anillos $ \Rightarrow  $  $ \theta ^{-1} : S \to R  $ también
    es un homomorfismo de anillos}
        \subsection{forma de demostracion}
            $ \theta a + b  $  = $ \theta a +  \theta b  $
            $ \theta a  b  $  = $ \theta a   \theta b  $

        \subsection{demostracion}
            supongamos que $ \varphi :R \to S  $  es un homomorfismo invertible de
            anillos , por lo tanto , si $ phi(p) = s \implies phi ^{-1} s = p $.
            \\ demostracion de que es un homomorfismo:
            \\$ \varphi ^{-1} a + \varphi ^{-1} b = \varphi ^{-1} a+b  $
            \\ como $ \varphi ^{-1} a \in R   $ y $ \varphi ^{-1} b \in R  $ entonces ..
            \\ $ \varphi(\varphi ^{1} a) + \varphi(\varphi ^{-1}  b)  = \varphi
            ^{-1} a + \varphi ^{-b}   $
            \\.



    \section{Encuentre el nucleo de los siguientes homomorfismos}
        \subsection{el nucleo es el kernel y es donde llegan todos los 0's}

    \begin{itemize}
        \item {        $ \theta :[x,y] \to R  $
    \\ $ f(x,y) \to f(0,0)  $
    \\ donde $ R[x,y]  $ es el anillo de los polinomios en dos indeterminadas $
    x  $ y $ y  $ , es decir: cada monomio  tiene la forma $ a_{ij}x ^{i} y
    ^{j}   $ con $ a_{ij} \in R $y $ i,j \in N  $}
    \item {$ \theta : R[x] \to C  $ ,
        \\ $ f(x) \to f(2+i)  $  }
    \end{itemize}

    sugerencia ,  encuentre el polinomo $ p(x)  $ de grado minimo en $
    ker(\theta )  $ y después muestre que cualquier elemento de $ ker(\theta )
    $ es de la forma $ p(x)q(x)  $   por algún $ q(x) \in R[x]  $ .







    %=======================NOTES ENDS HERE===================%

    % bib stuff
    \nocite{*}
    \addtocontents{toc}{\protect\vspace{\beforebibskip}}
    \addcontentsline{toc}{section}{\refname}
    \bibliographystyle{plain}
    \bibliography{../Bibliography}
\end{document}
