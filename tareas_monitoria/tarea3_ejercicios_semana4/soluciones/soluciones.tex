% article example for classicthesis.sty
\documentclass[10pt,a4paper]{article} % KOMA-Script article scrartcl
\usepackage{lipsum}     %lorem ipsum text
\usepackage{titlesec}   %Section settings
\usepackage{titling}    %Title settings
\usepackage[margin=10em]{geometry}  %Adjusting margins
\usepackage{setspace}
\usepackage{listings}
\usepackage{amsmath}    %Display equations options
\usepackage{amssymb}    %More symbols
\usepackage{xcolor}     %Color settings
\usepackage{pagecolor}
\usepackage{mdframed}
\usepackage[spanish]{babel}
\usepackage[utf8]{inputenc}
\usepackage{longtable}
\usepackage{multicol}
\usepackage{graphicx}
\graphicspath{ {./Images/} }
\setlength{\columnsep}{1cm}

% ====| color de la pagina y del fondo |==== %
\pagecolor{black}
\color{white}



\begin{document}
    %========================{TITLE}====================%
    \title{\rmfamily\normalfont\spacedallcaps{ Ejercicos monitoría semana 4 }}
    \author{\spacedlowsmallcaps{Rodrigo Castillo}}
    \date{\today}

    \maketitle


     % ====| Loguito |==== %
    \includegraphics[width=0.1\linewidth]{negro_cara.png}
    %=======================NOTES GOES HERE===================%
    \section{Sea $ R  $ un anillo, demuestre que }
        \subsection{si existe la identidad 1 es única}
        \subsection{si un elemento $ a  $ tiene inverso multiplicativo este es
        único}

    \section{Demuestre que los siguientes son subanillos de C}
        \subsection{Los enteros de Gauss}
            \begin{equation}
                Z[i] = \{a + bi ,a,b \in Z \}
            \end{equation}

        \subsection{Los enteros de einstein}
        \begin{equation}
                Z[w] = \{a + bw ,a,b \in Z \}
        \end{equation}
        donde $ w = e ^{ \frac{2 \pi i}{3} } = \frac{-1 + \sqrt{3} }{3}  $

    \section{Demuestre que si $ \theta : R \to S  $ es un homomorfismo
    invertible de anillos $ \Rightarrow  $  $ \theta ^{-1} : R \to S  $ también
    es un homomorfismo de anillos}

    \section{Encuentre el nucleo de los siguientes homomorfismos}
    \begin{itemize}
        \item {        $ \theta :[x,y] \to R  $
    \\ $ f(x,y) \to f(0,0)  $
    \\ donde $ R[x,y]  $ es el anillo de los polinomios en dos indeterminadas $
    x  $ y $ y  $ , es decir: cada monomio  tiene la forma $ a_{ij}x ^{i} y
    ^{j}   $ con $ a_{ij} \in R $y $ i,j \in N  $}
    \item {$ \theta : R[x] \to C  $ ,
        \\ $ f(x) \to f(2+i)  $  }
    \end{itemize}

    sugerencia ,  encuentre el polinomo $ p(x)  $ de grado minimo en $
    ker(\theta )  $ y después muestre que cualquier elemento de $ ker(\theta )
    $ es de la forma $ p(x)q(x)  $   por algún $ q(x) \in R[x]  $ .









    %=======================NOTES ENDS HERE===================%

    % bib stuff
    \nocite{*}
    \addtocontents{toc}{\protect\vspace{\beforebibskip}}
    \addcontentsline{toc}{section}{\refname}
    \bibliographystyle{plain}
    \bibliography{../Bibliography}
\end{document}
