% article example for classicthesis.sty
\documentclass[10pt,a4paper]{article} % KOMA-Script article scrartcl
\usepackage{import}
\usepackage{xifthen}
\usepackage{pdfpages}
\usepackage{transparent}
\newcommand{\incfig}[1]{%
    \def\svgwidth{\columnwidth}
    \import{./figures/}{#1.pdf_tex}
}
\usepackage{lipsum}     %lorem ipsum text
\usepackage{titlesec}   %Section settings
\usepackage{titling}    %Title settings
\usepackage[margin=10em]{geometry}  %Adjusting margins
\usepackage{setspace}
\usepackage{listings}
\usepackage{amsmath}    %Display equations options
\usepackage{amssymb}    %More symbols
\usepackage{xcolor}     %Color settings
\usepackage{pagecolor}
\usepackage{mdframed}
\usepackage[spanish]{babel}
\usepackage[utf8]{inputenc}
\usepackage{longtable}
\usepackage{multicol}
\usepackage{graphicx}
\graphicspath{ {./Images/} }
\setlength{\columnsep}{1cm}

% ====| color de la pagina y del fondo |==== %
\pagecolor{white}
\color{black}



\begin{document}
    %========================{TITLE}====================%
    \title{{  Ejercicios álgebra abstracta semana 7  }}
    \author{{Rodrigo Castillo}}
    \date{\today}

    \maketitle


    %=======================NOTES GOES HERE===================%
    \section{Sea $ R  $ un anillo con identidad, Demuestre}

        \subsection{El inverso de un elemento $ a  $ si existe es único}

            sea $ R  $ un anillo y $ b  $ un elemento en $ R  $ , sean $ b, b'
            $ elementos inversos de $ a '  $ , por lo tanto se tiene que $
            ba = 1  $ y que $ ba'  = 1 $ , así, se tiene lo siguiente:
            \\
            $ ab = 1  $
            \\
            $ ab' = 1  $
            \\
            $ ab = 1(ab') b' $
            \\
            $ ab = ab'  $
            \\
            $ b = b'  $
            \\
            por lo tanto solamente existe un elemento inverso para $ b  $

        \subsection{la identidad de 1 es una unidad y $ 1 ^{-1}  $  =1  }

            \subsubsection{la identidad de 1 es una unidad}
                sea $ 1  $ la identidad , por lo tanto se tiene que $ 1 \cdot 1
                $ =$ 1  $ , además, ya sabemos que la identidad es única, por
                lo tanto es una unidad
            \subsubsection{ $ 1 ^{-1} =1  $  }
                sea el elemento $ 1  $ , por lo tanto, existe un elemento $ a
                $ tal que $ 1 \cdot a  $  = 1, así, tenemos que
                \\
                $ 1 \cdot a = 1  $
                \\
                $ a = 1  $


        \subsection{si $ a  $ es una unidad, entonces $ a ^{-1}   $ tambien lo
        es y $ (a ^{-1})^{-1}    = 1  $ }

            sea $ a \in R  $ tal que $ a  $ es una unidad, es decir, que existe
            $ b \in  R  $ tal que $ a \cdot  b = 1  $ , de este modo, se tiene
            que  existe $ a  $ tal que $ b \cdot a = 1  $ , así, podemos decir
            que $ b  $ es una unidad en $ R  $ .

        \subsection{sean $ a,b  $ unidades, entonces $ ab  $ es unidad y $
        (ab)^{-1} = b ^{-1} a ^{-1}   $ }
            sean $ a,b \in R  $ tales que $ a,b  $ son unidades , de este modo,
            existen $ c,d  $ tales que $ ac = 1 , bd = 1  $ .
            \\
            $ ac = 1  $
            \\
            $ bd = 1  $
            \\
            $ ab \cdot cd = 1  $
            \\
            sea $ k  $  = $ cd  $ , entonces, se tiene $ k \in R  $ tal que $
            (ab)\cdot k = 1  $ , luego $ (ab)  $ es una unidad.


        \section{Sea $ R  $ un dominio de integridad, Demuestre que las
        unidades de $ R[x]  $  son exactamente los polinomios constantes
        que son unidades de $ R  $ también}

            supongamos que $ R  $ es un dominio de integridad , por lo tanto $
            R  $ es un anillo conmutativo que no tiene divisores de $ 0  $ .
            \\
            luego las unidades de $ R[x]  $ son $ i_1 , i_2 , i_3 ... i_x-1  $
            pues no hay divisores de $ 0  $ , por lo tanto son los polinomios
            constantes $ x_1 , x_2 , x_3 , ... x_n \in R  $  .


            \section{Encuentre las unidades del anillo $ Z[i]  $  }
            sea $ a+bi \in Z[i]  $  , por lo tanto para demostrar la unidad,
            tomaremos un $ a' b'i  $  tal que $ (a+bi) \cdot (a' b'i) =1  $
            \\
            por lo tanto :
            \\
            $ (a ^{2}  + b ^{2} ) \cdot (a' ^{2} + b' ^{2}  ) = 1  $
            \\
            $ a ^{2} + b ^{2 } = 1    $
            \\
            $ a = +- 1  $  y  $ b=0  $  o $ a = 0   $ y $ b = +-1  $
            \\
            $ a = +- i  $  y  $ b=0  $  o $ a = 0   $ y $ b = +-i  $
            luego los unicos elementos invertibles de $ Z[i]  $ son 1,-1 ,-i, i

            \section{punto 4 }
                Hecho






















    %=======================NOTES ENDS HERE===================%

    % bib stuff
    \nocite{*}
    \addtocontents{toc}{{}}
    \addcontentsline{toc}{section}{\refname}
    \bibliographystyle{plain}
    \bibliography{../Bibliography}
\end{document}
